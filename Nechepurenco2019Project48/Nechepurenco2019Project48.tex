\documentclass[12pt,twoside]{article}  
\usepackage{jmlda}
\usepackage[utf8]{inputenc}
\usepackage[russian]{babel}
\usepackage[T2A]{fontenc}
\usepackage{lineno}
\usepackage{setspace}

%\NOREVIEWERNOTES
\title
    [Образец оформления статьи для публикации] % Краткое название; не нужно, если полное название влезает в~колонтитул
    {Мультимоделирование, привилегированное обучение}
\author
    [Нечепуренко~И.\,О.] % список авторов для колонтитула; не нужен, если основной список влезает в колонтитул
    {Нечепуренко~И.\,О., Нейчев~Р.\,Г., Стрижов~В.\,В.} % основной список авторов, выводимый в оглавление
    [Нечепуренко~И.\,О.$^1$, Нейчев~Р.\,Г.$^2$, Стрижов~В.\,В.$^2$] % список авторов, выводимый в заголовок; не нужен, если он не отличается от основного
\thanks
    {Работа выполнена при финансовой поддержке РФФИ, проект \No\,00-00-00000.
   Научный руководитель:  Стрижов~В.\,В.
   Задачу поставил:  Эксперт~И.\,О.
    Консультант:  Консультант~И.\,О.}
\email
    {author@site.ru}
\organization
    {$^1$Организация; $^2$Организация}
\abstract
    {  Цель данной работы -  опробовать различные методы построения моделей оптимальной вычислительной сложности.  Затраты по времени и памяти для работы модели играют очень важную роль в большинстве сфер человеческой жизнедеятельности.  Носимая электроника и защищенные устройства для решения задач биометрии, устройства автоматической обработки телеметрических данных, системы потоковой аналитики результатов коллизий Большого Адронного Колайдера — это лишь малая доля случаев, когда  требуются достаточно быстрые алгоритмы машинного обучения. Существует широкиий спектр способов уменьшения сложности модели.  Один них - это разбиение объектов на подобласти,  и описание данных в каждой своей собственной моделью. Этот способ называется мультимоделированием (в иноязычной литературе Mixture experts).  Другой способ - мета-обучение (Metalearning),  метод, основанный на парадгме учителя-ученика, но при этом в качестве учителя могут выступать не только ответы на экспериментальных данных, но и результаты работы другой модели учителя. В случае добавления дополнительной, априорной информации, можно значительно улучшить результаты сходимость происходящего во время обучения оптимизационного процесса,  снизить сложность модели, а также повысить итоговое качество модели. Эти свойства были проверены на реальных данных.

\bigskip
\textbf{Ключевые слова}: \emph {машинное обучение, мета-обучение, мультимоделирование }.}
\titleEng
    {JMLDA paper example: file jmlda-example.tex}
\authorEng
    {Author~F.\,S.$^1$, CoAuthor~F.\,S.$^2$, Name~F.\,S.$^2$}
\organizationEng
    {$^1$Organization; $^2$Organization}
\abstractEng
    {This document is an example of paper prepared with \LaTeXe\
    typesetting system and style file \texttt{jmlda.sty}.

    \bigskip
    \textbf{Keywords}: \emph{keyword, keyword, more keywords}.}
\begin{document}
\maketitle
%\linenumbers
\section{Введение}

Рассматривается задача построения алгоритмов анализа данных максимальной точности с ограниченными затратами ресурсов. Конкретно работа посвящена самому распространенному типу машинного обучения - обучению с учителем. Обычно в промышлен ности и науке машинное обучение связанно с обработкой больших массивов данных и итерационными процессами оптимизации, а это влечет за собой сильные затраты по ресурсам, которые не могт себе позволить даже крупные компании. Рассматриваемая нами задача применима практически во всех сферах науки и бизнеса.

Существуют различные подходы к решению задачи. Один из них - мультимоделирование. Нам интересен один конкретный \cite{a}  алгоритм: смесь экспертов. В реализации этого алгоритма специальная функция, именуемая шлюзовой, опрееляет ценность предсказаний конкретного эксперта. Одной из полезных особенностей способа можно назвать возможность отфильтровывать "слабые" модели. 

Другой подход - модификация алгоритма обучения с учителем, в котором в роли учителя могут выступить не только экспериментальные данные, но и ответы машины, обученной на более полных данных либо более сложным алгоритмом.
Два метода, основанных на парадигме "машина учит машину":  дистилляция \cite{b}  и контроль сходства \cite{c} , были обобщены \cite{d} .

Рассматривваемые нами алгоритмы достаточно молодые: например,контроль сходства был представлен Вапником в 2009 г., а метод дистилляции - в 2015 -м году Д. Хинтоном.  Тем не менее, эти алгоритмы уже используются на практике: например, не так давно они использовались для использования энергии датацентрами Яндекса. 

Предлагается на реальных данных: данных о ценах энергопотребления в Польше в зависимости от времени протестировать известные алгоритмы, сравнить сложности вычисления и достигаемые качества при различных  реализациях. 

\section{Постановка задачи}

Наша задача  - разаработка моделей для решения классических задач обучения с учителем: как классификации, так и регрессии, а также более сложных. Итак, предполагается, что мы имеем набор объектов $\mathbb{X}$.  У каждого объекта есть набор признаков, принимающих действительные значения,  и лежащих в $\mathbb{R}^m$.  Такие значения можно задать матрицей 
$\mathbf{X} = [x_i]_{i = 1}^n$, где $x_i$ -  как раз вектор признаков $i$-го объекта. Также есть матрица ответов $\mathbf{Y} = [y_i]_{i = 1}^n$. Последняя, в завимости от задачи, может, в зависимости от постановки задачи, содежит метки классов, векторы значений и векторы распределений, $y_i \in \mathbb{R}^r$ Часто также обе матрицы объединяют в одну - $\hat{\mathbf{X}}$.

Наша цель - построение оптимальной модели $\hat{f}: \mathbb{R}^m \rightarrow \mathbb{R}^r$, с минимизирующую функцию ошибки $S(f,  \mathbf{X}, \mathbf{Y})$, $S$ принимает значения в $\mathbb{R}_+$, с ограничениями на сложность:
$$ \hat{f} = \argmin\limits_{f, |f| \leq M}{S(f, \mathbf{X}, \mathbf{Y})}$$ 
Точное определение понятия сложности, впрочем, выходит за рамки нашего исседования. 

\paragraph{Задача многоклассовой классификации.}
Положим $y_i \in \delta_r$, где $\delta_r$ - пространство векторов вероятности размерности $r$:
$$ y_i = [y_i^1, ..., y_i^r],$$
$$ \forall k: 0 \leq y_i^k  \leq 1,$$
$$ \sum\limits_{i = 1}^r y_i = 1$$	
Такая задача называется задачей классификации на $k$ классов, есть множество функций ошибки, мы будем использовать кросс-энтропию: 
$$S(y_i, \hat{f}(x_i)) =  - \sum\limits_{i=1}^r y_i^k \log \sigma(\hat{f}(x_i)^k), $$
$$\sigma (\hat{y})^k = \frac{\exp y^k}{\sum\limits_{k' = 1}^{r} \exp y^{k'}} $$
Функция $\sigma (\hat{y})^k $ также называется операцией softmax.

\paragraph{Задача декодирования}.
Если матрица ответов состоит из действительных векторов  $y_i \in \mathbb{R}^r$, то задачам декодирования соответствуют следующие функции ошибки:
$$MAE(y_i, \hat{f}(x_i)) = || y_i - \hat{f}(x_i) ||_1,$$
$$MSE(y_i, \hat{f}(x_i)) = || y_i - \hat{f}(x_i) ||_2,$$
$$MAPE(y_i, \hat{f}(x_i)) = ||\frac{ y_i - \hat{f}(x_i)}{y_i} ||_1,$$

\paragraph{Прогнозирование временных рядов как частный случай декодирования}.
Опредление:  временной ряд $s = [s_T, ... , s_i, .. , s_1]$ - последовательность наблюдений $s_i = s(t_i)$. (Выходит так, что время течет из настоящего в прошлое).

Также предположим, что нам дан набор из нескольких таких временных рядов $D = {s^q}$, $s \in \mathbb{R}^T$, $q =1, ..., Q$.
Каждому $s^q$ соответствует частота семплирования:  $\displaystyle \frac{1}{\tau^{(q)}} : t_i^{(q)} = i \cdot  \tau^{(q)}$. Сама же задача прогнозирования временных рядов звучит так: нам дается предыстория длиной $\delta t_p$, и из этого необходимо получить прогноз $\hat{s}$, где $[\hat{s}(t_i)]: T_{max} + \delta t_r \geq  t_i > T_{max}$, где $\delta t_r$ - некоторый промежуток времени, на который прогноз и делается.   Далее, в исходной формулировке можно сделать замену $y_i^{(q)} = [s^q(t_i), ... , s^q(t_i - \delta t_r)]$;
$x_i = [s^q(t_i - \delta t_r - 1), ... , s^q(t_i - \delta t_r - \delta t_p)]$. Мы можем построить матрицу плана $\hat{\mathbf{X}}$, выбрав множество моментов разбиения ${t_i}, i = 0 ... (n) $ так, что сегменты $s_i^q = [y_i| x_i]$, покрывающие временной ряд $s^q$, были упорядочены: $t_{i + 1} > t_i \forall i$, и разбив каждый из рядов ${s^q}$.

\paragraph{Процедура скользящего контроля для временных рядов}.
Процедура скользящего контроля - один из методов проверки адекватности модели $\hat{f}$ на базе исторических данных. На самом деле, это стандартный алгоритм кросс-валидации, только с учетом специфики случайных процессов.
В рамках этой процедуры рассматривается $V$ сегментов времени, упорядоченных хронологически. Каждый из сегментов имеет фиксированную длину $\delta_b$, начинается во время $t_B$ и соответствует матрице плана $\hat{ \mathbf{X}}_b$.

	Алгоритм:
\begin{enumerate}
  \item Фиксируется некоторое семейство функций $\mathbb{F}$, среди которых и ищется оптимальная модель. Также полагается $b = 0$
  \item Первой строкой матрицы плана $\hat{X}_b$ положим пару векторов $y_{val, b}, x_{val, b}$, соответствующую промежутку длиной $\delta t_r$
  \item Теперь дополним матрицу локальной предысторией $\hat{\mathbf{X}}_{train, b} =  [ \mathbf{Y}_{train, b},  \mathbf{X}_{train, b}]$, соответствующей промежутку $\delta t_B = \delta t_r$: это будет 

   \item $\hat{f}$ ищется как решение исходной задачи минимизации на подпространстве $\hat{\mathbf{X}}_{train, b}$.

 \item Ошибка оценивается на $[y_{val, b}, x_{val, b}]$

\item Повторение иттерации с пункта 2
\end{enumerate}




\section{Название раздела}
Данный документ демонстрирует оформление статьи,
подаваемой в электронную систему подачи статей \url{http://jmlda.org/papers} для публикации в журнале <<Машинной обучение и анализ данных>>.
Более подробные инструкции по~стилевому файлу \texttt{jmlda.sty}
и~использованию издательской системы \LaTeXe\
находятся в~документе \texttt{authors-guide.pdf}.
Работу над статьёй удобно начинать с~правки \TeX-файла данного документа.

\paragraph{Название параграфа.}
%Первый раздел может содержать формальную постановку задачи,
%основные определения и~обозначения,
%известные факты, необходимые для понимания основных результатов работы,
%и~т.\,п.
Нет ограничений на~количество разделов и~параграфов в~статье.
Разделы и~параграфы не~нумеруются.


\section{Заключение}
Желательно, чтобы этот раздел был, причём он не~должен дословно повторять аннотацию.
Обычно здесь отмечают,
каких результатов удалось добиться,
какие проблемы остались открытыми.



\bibliography{Nechepurenco2019Project48}{}
\bibliographystyle{plain}

% Решение Программного Комитета:
%\ACCEPTNOTE
%\AMENDNOTE
%\REJECTNOTE
\end{document}

